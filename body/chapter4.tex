\section{脚手架工程专项安全施工方案}
\subsection{编制依据}

脚手架的搭设规范主要参考自以下规范或文件:

(1) 《建筑施工脚手架使实用手册》

(2) 《建筑施工扣件式钢管脚手架安全技术规范》(JGJ130-2011)

(3) 《建筑施工高处作业安全规范》(JGJ80-2016)

(4) 《钢结构工程施工规范》(GB50755-2012)

(5) 《建筑结构荷载规范》(GB50009-2012)

(6) 《建筑施工手册》

(7) 《建设工程安全生产管理条例》(国务院令第 393 号)

\subsection{脚手架搭设要求}
\subsubsection{落地式脚手架搭设要求}

\subsubsection{悬挑式脚手架搭设要求}

\subsection{方案选择}

\subsection{主要参数}

\subsection{脚手架计算书}
\subsubsection{双排脚手架安全性验算}

(1) 小横杆计算

按照 JGJ130-2011,小横杆按照简支梁进行强度和挠度计算,用大横杆支座的最大反力计算值作为小横杆集中荷载,在最不利荷
载布置下计算小横杆的最大弯矩和变形。\\

\quan{1} 均布荷载计算\\

小横杆的自重标准值:
\begin{align}
    p_1=0.0397 \times 1=0.0397 kN/m
\end{align}

脚手板的自重标准值:
\begin{align}
    p_2=\frac{0.35 \times 1.5}{2+1}=0.175 kN/m
\end{align}

施工荷载自重标准值:
\begin{align}
    Q=\frac{3 \times 1.5}{2+1}=1.5 kN/m
\end{align}

恒荷载做控制荷载设计值:
\begin{align}
    q_1=1.35\times(0.0397+0.175)+1.4\times 0.7\times 1.5=1.759 kN/m
\end{align}

活荷载做控制荷载设计值:
\begin{align}
    q_2=1.2\times(0.0397+0.175)+1.4\times 1.5=2.396 kN/m
\end{align}

\quan{2} 强度验算\\

恒荷载、活荷载二者取最大值,得出由活荷载做控制 $q=2.396 kN/m$。由于小横杆按照简支梁计算,故跨中弯矩值最大;
\begin{align}
    M_{max}=\frac{ql^2}{8}=\frac{2.396\times 1.05^2}{8}=0.33 kN\cdot m
\end{align}
式中:$q$ 为小横杆荷载设计值;$l$ 为小横杆计算跨度,即立杆横距。

计算最大应力,其中 $M_max$ 为最大弯矩, $W$ 为截面模量,取 $5.26 cm^3$,可得:
\begin{align}
    \sigma =\frac{M}{W}=\frac{0.33\times 10^6}{5.26\times10^3}=173.58 N/mm^2
\end{align}
小横杆的计算强度 $173.58N/mm^2$ 小于小横杆的抗弯强度设计值 $205N/mm^2$,故强度满足要求!\\

\quan{3} 挠度验算\\

水平杆的挠度验算应满足 $v\leq [v] $,其中 $v$ 是挠度;小横杆的挠度计算式为:
\begin{align}
    V=\frac{5ql^4}{384EI}=\frac{5\times (0.397+0.175)\times 1050\times 10^4}{384\times 20.6\times 10^5\times 121870}=0.36 mm
\end{align}
式中 $E$ 为弹性模量,取$2.06\times 10^5$,$I$ 为惯性矩。

小横杆的最大挠度 $0.36mm$ 小于 $1050.0/150=7.000$ 与 $10mm$,故挠度满足要求! \\

(2) 大横杆计算

(3) 扣件抗滑计算

(4) 立杆荷载计算

(5) 连墙件计算

(6) 地基承载力计算

\subsection{脚手架质量验收}

(1) 脚手架搭设前,对进入现场的各种构配件应按下列规定进行检查验收,不合格的应及时清除出场;
构配件应有相应的产品标识及产品质量合格证;构配件应有相应的技术参数及产品使用说明书;当对构配件质量有疑问时,应进行质量抽检和实
验。

(2) 脚手架在悬挑层顶梁板浇筑后及脚手架搭设前;作业层上施加荷载前;每搭设完两层后;达到设计高度后;遇有六级强风及以上风或大雨后;停
用超过一个月时应该进行检查与验收,按规定对脚手架工程的质量进行检查,合格后方可交付使用;

(3) 架子搭设和组装完毕,使用前必须由项目经理、技术负责人、项目安全负责人、架子班长等人员组成验收小组,
进行验收,并填写验收单。

(4) 脚手架使用期间必须设专人经常检查,符合要求后,必须经过项目经理签字批准,才能使用;
不合格部位必须及时修复或更换,符合规定后,方准许继续使用。 

\subsection{安全技术措施}
\subsubsection{脚手架搭设安全技术措施}

\subsubsection{脚手架拆除安全技术措施}
