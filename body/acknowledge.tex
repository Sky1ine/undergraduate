\begin{center}
    \section*{\zihao{-2}  \textbf{致 ~~ 辞}}
    \end{center}

随着论文的结尾,也标志这我的大学四年生活正式画上了句号,我也将与这所坐落在沈城的美丽的学校即将挥手告别,在
大学四年的学习与生活之中有许许多多帮助过我的,给予我关怀的人,他们有些人是我的长辈,有些人是我的老师,有些人是我的同学,甚至有些事,有些景
也都在不同时间、以不同的方式给我加油、给我在或是志得意满,或是人生低谷时帮助我端正心态,调整方向,在此再次深表感谢和感激。

大学四年来,首先要感谢的是我的父母,是他们用无私的爱、无私的关怀以及毫无保留的生活和经济支持不计回报的支撑着我走完大学四年的学习生活,可以说
在我十五载求学生涯之中是他们给了我最大的鼓励和支持,感谢父母对我的理解和包容,也感谢他们无私奉献的爱。对父母的恩情是无论如何也报答不完的,愿我能
在余生之内始终伴随我的父母,希望他们身体健康,平平安安。

其次要感谢的是土木学院安全教研室的全体老师,我本身算不上是聪明用功的学生,感谢这些老师们不厌其烦的教导,以及在学习生活中给予我们的关怀和爱护,在
建筑大学这个小世界里,我们真真正正感受到了来着“直系老师”们的关爱,让我们在离开家庭的保护之后还能感受到充满人文的,充满老师们独有的来自长辈的关心。
在这里也祝福土木学院安全教研室贾老师在内的全体老师们万事顺遂。

我还要感谢曾经帮助过我的学长们,是你们将懵懂无知的我从一个刚离开家门的完全不自立的孩子变成了一个懂得学会付出,懂得学会与他人合作,懂得为自己的一言一行而负责,
这里特别是要感谢我的助班谢建坤,感谢他在我刚刚步入校门时给予我兄长一般的关怀;其次要感谢我在大学生通讯社认识的部长们,虽然我未能陪大家走完大学学生工作的
最后一程,但是我会永远记住你们对我的包容和支持,你们教会我的处事之道和协作之力,也希望所有帮助过和希望帮助过我的学长学姐们学业有成、前程似锦、心想事成。

大学生活是我头一次离开家中,与同学一起生活,一起居住,在这里我要感谢我的室友,在四年的日子里与我共同进退,共渡难关,在未来的日子里,我将无时不刻不在怀念你们与我共同
生活这几年里的点点滴滴,更感谢你们能包容我糟糕的性格和坏习惯,祝我的室友们能早日找到自己的另一半,早日过上自己所希望的生活。

除了这些常规的感谢之外,我还要感谢那些素未谋面但是却亲如兄弟的人们,感谢我“图鉴”项目中的全体成员,感谢你们在我需要鼓励时给予我鼓励,在我需要安慰时给予我安慰,感谢岳君宇同学对我文章语意连贯性的修改,
特别是要感谢谈梦飞同学,感谢你让我更好的认识我自己,认识这个社会,感谢你让幼稚的我学着成熟,学着承担。

2020 年注定是艰难的一年,也注定是充满挑战的一年,很难想象我们这代人居然以这种方式见证了历史。我们经历了一个没有团圆的春节、一个没有毕业照的毕业季和一个没有毕业典礼的大学。但我们
没有害怕,更没有放弃,因为我们相信迷雾之后必将是阳光灿烂,黑暗之后必是满山黎明,等到疫情结束之时,就是我们这代人的胜利之日,最后我要感谢的是全社会的努力,当然也包括
感谢正在工作的你,或是感谢正在发呆的我。正是因为有了大家的齐心协力,我们才蹒跚却坚定地走完了 2020 年的上半年,在接下来的日子里,大家也要满怀希望啊!长风破浪的日子,不会太远啦!

在全篇致辞的最后,我想用一句话结束我的本科生毕业论文,也想用这一句话来勉励我今后的生活。\\

天下的所有坎坷之事终将消散,唯有爱与希望将永世长存。\\

共勉。
\addcontentsline{toc}{section}{致辞}
